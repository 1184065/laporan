\documentclass[12pt]{ociamthesis}  % default square logo 
%\documentclass[12pt,beltcrest]{ociamthesis} % use old belt crest logo
%\documentclass[12pt,shieldcrest]{ociamthesis} % use older shield crest logo

%load any additional packages
\usepackage{hyperref}
\usepackage{amssymb}
\usepackage{float}
\usepackage{amsmath}
\usepackage{longtable}
%\usepackage{listings} %code listing, memasukkan code
\usepackage{listings}
\usepackage{xcolor}
 
\definecolor{codegreen}{rgb}{0,0.6,0}
\definecolor{codegray}{rgb}{0.5,0.5,0.5}
\definecolor{codepurple}{rgb}{0.58,0,0.82}
\definecolor{backcolour}{rgb}{0.95,0.95,0.92}
 
\lstdefinestyle{mystyle}{
    backgroundcolor=\color{backcolour},   
    commentstyle=\color{codegreen},
    keywordstyle=\color{magenta},
    numberstyle=\tiny\color{codegray},
    stringstyle=\color{codepurple},
    basicstyle=\ttfamily\footnotesize,
    breakatwhitespace=false,         
    breaklines=true,                 
    captionpos=b,                    
    keepspaces=true,                 
    numbers=left,                    
    numbersep=5pt,                  
    showspaces=false,                
    showstringspaces=false,
    showtabs=false,                  
    tabsize=2
}
 
\lstset{style=mystyle}
%input macros (i.e. write your own macros file called mymacros.tex 
%and uncomment the next line)
%\include{mymacros}

\title{Aplikasi Scan Barcode Barang Masuk dan Keluar Berbasis Android\\[3ex]\textit{logbook}     %your thesis title,
}   %note \\[1ex] is a line break in the title

\author{Etika Khusnul Laeli\\1.18.4.065\\Jenly Ramda\\1.18.4.109}             %your name
\college{}  %your college

%\renewcommand{\submittedtext}{change the default text here if needed}
\degree{Applied Bachelor Program of Informatics Engineering}     %the degree
\degreedate{Bandung\\ 2019}         %the degree date

%end the preamble and start the document
\begin{document}

%this baselineskip gives sufficient line spacing for an examiner to easily
%markup the thesis with comments
\baselineskip=18pt plus1pt

%set the number of sectioning levels that get number and appear in the contents
\setcounter{secnumdepth}{3}
\setcounter{tocdepth}{3}


\maketitle                  % create a title page from the preamble info
%\include{section/acknowlegements}   % include an acknowledgements.tex file
%\include{section/abstract}          % include the abstract

\begin{romanpages}          % start roman page numbering
\tableofcontents            % generate and include a table of contents
%\listoffigures              % generate and include a list of figures
\end{romanpages}            % end roman page numbering

%now include the files of latex for each of the chapters etc
\chapter{Pertemuan 1}

\section{Issues \#1}
Pada \textit{issues \#1} (\textit{Hardcoded text}) Keterangan dan pemechan masalah : ini hanya peringatan, fungsi tetap bekerja sebagai mana mestinya. direkomendasikan untuk mendefine string "text" di file string.xml. alasannya agar membuat perubahan di masa depan lebih sederhana dan lebih mudah untuk referensi di beberapa halaman.
\\lokasi
\begin{verbatim}
  <Button
  android:id="@+id/buttonScan"
  android:layout_width="match_parent"
  android:layout_height="wrap_content"
  android:layout_alignParentBottom="true"
  android:text="Scan QR Code" />       
\end{verbatim}
sehingga solusinya didapatkan dengan cara mendownload packages python pyAudio yang sesuai dengan versi python yang digunakan. Setelah itu install melalui cmd dengan mengetikkan \begin{verbatim}
  <resources>
  <string name="app_name">QRCodeScanner</string>
  </resources>
\end{verbatim}
Pada merupakan error yang terjadi di saat hendak melakukan instalasi PyAudio melalui pip install PyAudio\begin{verbatim}
  <TextView
    android:layout_width="match_parent"
    android:layout_height="wrap_content"
    android:text="@string/string_name"/>
\end{verbatim}
Pada merupakan error yang terjadi di saat hendak melakukan instalasi PyAudio melalui pip install PyAudio\begin{verbatim}
  <resources>
    <string name="app_name">QRCodeScanner</string>
    <string name="scan_qr_code">Scan Qr Code</string>
  </resources>
\end{verbatim}

\section{Issues \#2}
Pada \textit{issues \#2} (\textit{MissingPrefix}) yaitu Error yang terjadi ketika xxxx.
\\lokasi :
\begin{verbatim}
  <Button
  android:id="@+id/buttonScan"
  android:layout_width="match_parent"
  android:layout_height="wrap_content"
  android:layout_alignParentBottom="true"
  android:text="Scan QR Code" />       
\end{verbatim}
Perbaikan kode progam :\begin{verbatim}
  <TextView
    android:layout_width="match_parent"
    android:layout_height="wrap_content"
    android:text="@string/string_name"/>
\end{verbatim}

\section{Issues \#3}
Pada \textit{issues \#2} (\textit{Jenis Error : identifier expected}) yaitu Error yang terjadi ketika xxxx.
\\lokasi :
\begin{verbatim}
  protected void onPostExecute(s) {
     super.onPostExecute(s);
     loading.dismiss();
     Toast.makeText(MainActivity.this,s,Toast.LENGTH_LONG).show();
  }
\end{verbatim}
Perbaikan kode progam :\begin{verbatim}
  protected void onPostExecute(String s) {
     super.onPostExecute(s);
     loading.dismiss();
     Toast.makeText(MainActivity.this,s,Toast.LENGTH_LONG).show();
  }
\end{verbatim}

\section{Issues \#4}
Pada \textit{issues \#2} (\textit{Jenis Error : variable ‘x’ is already defined in scope})
\\lokasi :
\begin{verbatim}
  RequestHandler rh = new RequestHandler();
  String rh = rh.sendPostRequest(konfigurasi.URL_ADD, params);
  return rh;
\end{verbatim}
Keterangan dan splusi :
\\tidak dapat memanggil variabel sendidiri dan membuat variabel lokal dengan nama yang sama di java. solusi yang dapat dilakukan adalah mengganti nama salah variable
Perbaikan kode progam :\begin{verbatim}
  protected void onPostExecute(String s) {
     super.onPostExecute(s);
     loading.dismiss();
     Toast.makeText(MainActivity.this,s,Toast.LENGTH_LONG).show();
  }
\end{verbatim}

\section{Issues \#5}
Pada \textit{issues \#2} (\textit{Jenis Error : Incompatible Types})
\\lokasi :
\begin{verbatim}
  buttonScan = (“Button”) findViewById(R.id.buttonScan);
  buttonAdd = (“Button”) findViewById(R.id.buttonAdd);
  buttonTabel = (“Button”) findViewById(R.id.buttonTabel);
\end{verbatim}
Keterangan dan pemecahan masalah :
\\ini merupakan sebuah error pada MainActivity.java karena penggunaan dalam syntak salah yaitu dengan menggunakan tanda kutip pada kode Button. adalah mengganti nama salah variable
Perbaikan kode progam :\begin{verbatim}
  buttonScan = (Button) findViewById(R.id.buttonScan);
  buttonAdd = (Button) findViewById(R.id.buttonAdd);
  buttonTabel = (Button) findViewById(R.id.buttonTabel);
\end{verbatim}

\section{Issues \#6}
Pada \textit{issues \#2} (\textit{warning Field can be converted to a local variable Lokasi File MainActivity.java})
\\lokasi :
Keterangan dan pemecahan masalah :
\\ini merupakan sebuah warning atau peringatan pada MainActivity.java karena
pada IntentIntegrator kelebihan dalam bidang kelasnya dan jika kelebihan bisa diganti dengan variable lokal.
\end{verbatim}

\section{Issues \#7}
Pada \textit{issues \#2} (\textit{Local variable 'res' is redundant less Lokasi File MainActivity.java})
\\lokasi :
\begin{verbatim}
String res = rh.sendPostRequest(konfigurasi.URL_ADD, params);
\end{verbatim}
\\ini merupakan sebuah warning atau peringatan pada MainActivity.java, warning tersebut meliputi; variabel yang ditangkap meliputi variabel lokal yang segera dikembalikan, variabel lokal yang segera ditugaskan ke variabel lain dan kemudian tidak digunakan, dan variabel lokal yang selalu memiliki nilai yang sama dengan variabel atau parameter lokal lainnya.


\section{Issues \#8}
Pada \textit{issues \#2} (\textit{Cannot resolve symbol ‘parent’})
\\lokasi :
protected void onActivityResult(int requestCode, int resultCode, Intent data) {
IntentResult result = IntentIntegrator.parseActivityResult(requestCode, resultCode, data);
if (result != null)
\\Keterangan dan Solusi:
Library belum diimport solusinya dengan cara mengimport library jenis zxying:
\begin{verbatim}
import com.google.zxing.integration.android.IntentIntegrator;
import com.google.zxing.integration.android.IntentResult;
\end{verbatim}

\section{Issues \#9}
Pada \textit{issues \#2} (\textit{taticFieldLeak})
\\lokasi : Lokasi File MainActivity.java
protected void onActivityResult(int requestCode, int resultCode, Intent data) {
IntentResult result = IntentIntegrator.parseActivityResult(requestCode, resultCode, data);
if (result != null)
\end{verbatim}
\\Keterangan dan pemecahan masalah:
AsyncTask sebagai class non static inner yang akan menerima reference dari parent class. untuk memperbaikinya teruskan aktivity yang lemah ke constructor AsyncTask. kemudian kamu dapat memanggil intent code di onPostExecute.

\section{Issues \#10}
Pada \textit{issues \#10} (\textit{XXXXXXXXXXX)}) fungsi yang digunakan untuk memberikan \textit{Exception} apabila program gagal mendeteksi suara atau terjadi kesalahan atau \textit{error} yang kemudian akan dicetak \textit{error} tersebut dan mencetak "TIME OVER, THANK YOU"
\chapter{Pertemuan 2}

\section{Issues \#11}
Pada \textit{issues \#11} (\textit{from selenium import webdriver}) merupakan fungsi yang digunakan untuk mengimportkan webdriver dari modul selenium
\begin{verbatim}
from selenium import webdriver
\end{verbatim}
\section{Issues \#12}
Pada \textit{issues \#12} (\textit{from selenium.webdriver.firefox.options import Options}) 
digunakan untuk mengimport Options dari webdriver firefox dari modul selenium
\begin{verbatim}
from selenium.webdriver.firefox.options import Options
\end{verbatim}
\section{Issues \#13}
Pada \textit{issues \#13} (\textit{opsi = Options()}) 
untuk membuat variable opsi dengan value Options() dari modul selenium yang telah diinstallkan pada program di PC kita
\begin{verbatim}
opsi = Options()
\end{verbatim}
\section{Issues \#14}
Pada \textit{issues \#14} (\textit{opsi = webdriver.firefox.options.Options()}) ialah variabel yang digunakan untuk mengambil value Options() dari webdriver firefox yang digunakan
\section{Issues \#15}
Pada \textit{issues \#15} (\textit{opsi.headless = False}) merupakan kode yang digunakan untuk menonaktifkan \textit{headless mode} pada firefox browser yang kita gunakan
\begin{verbatim}
opsi.headless = False
\end{verbatim}
\section{Issues \#16}
Pada \textit{issues \#16} (\textit{cap = webdriver.common.desired\_capabilities.DesiredCapabilities().FIREFOX}) variabel yang berfungsi untuk memberitahukan pada webdriver environtment mana yang akan digunakan pada \textit{script}.
\section{Issues \#17}
Pada \textit{issues \#17} (\textit{cap['marionette'] = True}) 
marionette digunakan untuk driver otomatisasi pada gecko mozilla firefox
\begin{verbatim}
cap['marionette'] = True
\end{verbatim}
\section{Issues \#18}
Pada \textit{issues \#18} (\textit{fungsionalitas driver = webdriver.Firefox()}) merupakan variabel yang digunakan untuk memulai suatu webdriver berbasiskan Firefox
\begin{verbatim}
driver = webdriver.Firefox()}
\end{verbatim}
\section{Issues \#19}
Pada \textit{issues \#19} (\textit{fungsionalitas driver.get("https://siap.poltekpos.ac.id")}) 
berfungsi untuk mendapatkan akses ke situs yang dituju melalui driver yang digunakan
\begin{verbatim}
driver.get("https://siap.poltekpos.ac.id")
\end{verbatim}
\section{Issues \#20}
Pada \textit{issues \#20} (\textit{fungsionalitas driver.find\_element\_by\_name('user\_name').send\_keys("1184030")}) digunakan untuk memungkinkan driver mendapatkan element username dengan melalui nama dan kemudian mengirimkan inputannya
\begin{verbatim}
self.driver.find_element_by_name('user_name').send_keys("1184030")
\end{verbatim}
\chapter{Pertemuan 3}
\section{Issues \#21}
Pada \textit{issues \#21} (\textit{class Apem(object):}) ialah pembuatan class sehingga program dapat dijalankan secara \textit{Object-Oriented-Programming} / OOP 
\begin{verbatim}
class Apem(object): 
\end{verbatim}
\section{Issues \#22}
Pada \textit{issues \#22} (\textit{def \_\_init\_\_(self, filname): self.filname = filname}) pembuatan constructor dengan parameter self dan filname
\section{Issues \#23}
Pada \textit{issues \#23} (\textit{def masuk(self):}) ialah fungsi yang berisikan method yang akan menampilkan tampilan halaman sistem akademik siap di Politeknik Pos Indonesia
\begin{verbatim}
def masuk(self): 
        self.opsi = Options()
        self.opsi.headless = False
        self.cap = webdriver.common.desired_capabilities.DesiredCapabilities().FIREFOX
        self.cap['marionette'] = True
        self.driver= webdriver.Firefox()
        self.driver.get('http://siap.poltekpos.ac.id/')
\end{verbatim}
\section{Issues \#24}
Pada \textit{issues \#24} (\textit{def login(self):}) fungsi yang berisikan method yang akan menampilkan tampilan halaman login ke sistem akademik siap di Politeknik Pos Indonesia
\begin{verbatim}
 def login(self): 
        self.opsi = Options()
        self.opsi.headless = False
        self.cap = webdriver.common.desired_capabilities.DesiredCapabilities().FIREFOX
        self.cap['marionette'] = True
        self.driver= webdriver.Firefox()
        self.driver.get('http://siap.poltekpos.ac.id/siap/besan.depan.php')
        self.driver.find_element_by_name('user_name').send_keys(self.npm)
        self.driver.find_element_by_name('user_pass').send_keys(self.paswd)
        self.driver.find_element_by_xpath('/html/body/table/tbody/tr[5]/td/table[1]/tbody/tr/td[2]/table[2]/tbody/tr[1]/td[2]/div/form/input[4]').click()
\end{verbatim}
\section{Issues \#25}
Pada \textit{issues \#25} (\textit{self.driver.find\_element\_by\_xpath('/html/body/table/tbody/tr[5]/td/table[1]/tbody/tr/td[2]/table[2]/tbody/tr[1]/td[2]/div/form/input[4]').click()}) merupakan variabel yang memungkinkan driver untuk mendapatkan element pada button login melalui xpath kemudian program akan secara otomatis melakukan method click pada button tersebut
\section{Issues \#26}
Pada \textit{issues \#26} (\textit{def speak(self):}) fungsi yang dibuat untuk memberikan method yang terjadi ketika user mengucapkan sesuatu ke program kemudian program menangkap audio yang diterima dengan menggunakan \textit{try and except}
\begin{verbatim}
def speak(self):
        r= sr.Recognizer()
        with sr.Microphone() as source:
            print("SAY SOMETHING, PLEASE")
            audio = r.listen(source)
        try:
            print("TEXT : "+r.recognize_google(audio, language='id-ID'))
            x = "siap"
            y = "login siap"
            z = "Cek nilai semester 1"
            a = "Cek nilai semester 2"
            b = "Cek nilai semester 3"
            c = "Cek nilai semester pendek"
            d = "kalender akademik ganjil 2017"
            e = "kalender akademik ganjil 2018"
            f = "kalender akademik ganjil 2019"
            g = "kalender akademik genap 2017"
            h = "kalender akademik genap 2018"
            i = "kalender akademik genap 2019"
            j = "jadwal semester 3"
            k = "jadwal semester 2"
            l = "jadwal semester 1"
            m = "nomor saya"
            n = "alamat saya"
            o = "nomor ortu"
            p = "mail ortu"
            if (r.recognize_google(audio, language='id-ID')) == x:
                self.masuk()
            if (r.recognize_google(audio, language='id-ID')) == y:
                self.login()
            if (r.recognize_google(audio, language='id-ID')) == z:
                self.ceknilai1()
            if (r.recognize_google(audio, language='id-ID')) == a:
                self.ceknilai2()
            if (r.recognize_google(audio, language='id-ID')) == b:
                self.ceknilai3()
            if (r.recognize_google(audio, language='id-ID')) == c:
                self.ceknilaipendek()
            if (r.recognize_google(audio, language='id-ID')) == d:
                self.kalenderganjil2017()
            if (r.recognize_google(audio, language='id-ID')) == e:
                self.kalenderganjil2018()
            if (r.recognize_google(audio, language='id-ID')) == f:
                self.kalenderganjil2019()
            if (r.recognize_google(audio, language='id-ID')) == g:
                self.kalendergenap2017()
            if (r.recognize_google(audio, language='id-ID')) == h:
                self.kalendergenap2018()
            if (r.recognize_google(audio, language='id-ID')) == i:
                self.kalendergenap2019()
            if (r.recognize_google(audio, language='id-ID')) == j:
                self.jadwalsmt1()
            if (r.recognize_google(audio, language='id-ID')) == k:
                self.jadwalsmt2()
            if (r.recognize_google(audio, language='id-ID')) == l:
                self.jadwalsmt3()
            if (r.recognize_google(audio, language='id-ID')) == m:
                self.nomerhp()
            if (r.recognize_google(audio, language='id-ID')) == n:
                self.alamatsiswa()
            if (r.recognize_google(audio, language='id-ID')) == o:
                self.nomerortu()
            if (r.recognize_google(audio, language='id-ID')) == p:
                self.emailortu()
\end{verbatim}
\section{Issues \#27}
Pada \textit{issues \#27} (\textit{print("Time is over, thanks")}) merupakan method berupa mencetak "Time is over, thanks" yang akan dijalankan ketika program telah selesai dieksekusi
\begin{verbatim}
print("Time is over, thanks")
\end{verbatim}
\section{Issues \#28}
Pada \textit{issues \#28} (\textit{import apem}) merupakan perintah yang digunakan untuk mengimportkan file apem
\begin{verbatim}
import apem
\end{verbatim}
\section{Issues \#29}
Pada \textit{issues \#29} (\textit{speech = apem.Apem(object)}) merupakan variabel yang berisikan perintah untuk penginisiasian objek dari file apem di \textit{class} Apem.
\begin{verbatim}
speech = apem.Apem(object)
\end{verbatim}
\section{Issues \#30}
Pada \textit{issues \#30} (\textit{speech.speak()}) merupakan perintah yang digunakan untuk menjalankan method pada fungsi speak() yang telah dibuat pada class Apem sebelumnya.
\begin{verbatim}
speech.speak()
\end{verbatim}
\chapter{Pertemuan 4}

\section{Issues \#31}
Pada \textit{issues \#31} (\textit{def \_\_init\_\_(self, npm, paswd):}) merupakan constructor yang berisikan parameter berupa npm dan paswd sehingga kedua variabel tersebut adalah variabel yang dieksekusi pertama kali saat program dijalankan
\begin{verbatim}
def __init__(self, npm, paswd):
        self.npm = npm
        self.paswd = paswd
\end{verbatim}

\section{Issues \#32}
Pada \textit{issues \#32} (\textit{self.npm = npm self.paswd = paswd}) merupakan pendeklarasian variabel yang ada pada constructor
\begin{verbatim}
self.npm = npm
self.paswd = paswd
\end{verbatim}

\section{Issues \#33}
Pada \textit{issues \#33} (\textit{self.driver.find\_element\_by\_name('user\_name').send\_keys(self.npm)}) fungsi ini digunakan untuk memungkinkan driver untuk mengambil elemen menggunakan nama user\_name dan mengirimkan inputan yang didapatkan dari variabel npm

\section{Issues \#34}
Pada \textit{issues \#34} (\textit{self.driver.find\_element\_by\_name('user\_pass').send\_keys(self.paswd)})
fungsi ini digunakan untuk memungkinkan driver untuk mengambil elemen menggunakan nama user\_pass dan mengirimkan inputan yang didapatkan dari variabel paswd

\section{Issues \#35}
Pada \textit{issues \#35} (\textit{self.driver.find\_element\_by\_name('login').click()}) fungsi ini digunakan untuk memungkinkan driver untuk mengambil elemen menggunakan nama "login" dan melakukan method berupa click elemen tersebut

\section{Issues \#36}
Pada \textit{issues \#36} (\textit{Fungsionalitas npm=input("Masukkan NPM")}) merupakan pendeklarasian variabel npm yang dilakukan dengan melakukan inputan user

\section{Issues \#37}
Pada \textit{issues \#37} (\textit{Fungsionalitas paswd=input("Masukkan password akun SIAP anda")}) 
merupakan pendeklarasian variabel paswd yang dilakukan dengan melakukan inputan user
\section{Issues \#38}
Pada \textit{issues \#38} (\textit{Fungsionalitas TypeError: \_init\_() missing 1 required positional argument: 'paswd'}) merupakan \textit{error} yang disebabkan oleh adanya memasukkan parameter yang tidak sesuai dengan \textit{constructor}
\begin{verbatim}
speech = apem.Apem(object,npm)#issue 38
\end{verbatim}
pada kode tersebut, penginisiasian objectnya tidak sesuai karena parameternya salah, seharusnya parameter yang digunakan ialah (npm, paswd), sehingga sebagai berikut :
\begin{verbatim}
speech = apem.Apem(npm, paswd) #penyelesaian issue 38
\end{verbatim}
\section{Issues \#39}
Pada \textit{issues \#39} (\textit{Too many positional arguments for method call}) merupakan error yang disebabkan karena penempatan argument yang sebernarnya tidak diperlukan
\begin{verbatim}
speech.speak(object)#issue 39 
\end{verbatim}
kemudian berikut ini merupakan penyelesaian yang digunakan :
\begin{verbatim}
speech.speak()#penyelesaian issue 39
\end{verbatim}
\section{Issues \#40}
Pada \textit{issues \#40} (\textit{Message: Reached error page: about:neterror?e=connectionFailure&u=https\%3A\\siap.poltekpos.ac.id/&c=UTF-8&f=regular&d=Firefox\%20can\%E2\%80\%99t\%20establish\%20a\%20connection\%20to\%20the\%20server\%20at\%20siap.poltekpos.ac.id.}) error tersebut disebabkan karena adanya program dari driver yang tidak sesuai, sehingga harus dihilangkan. Berikut ini merupakan kode yang dihilangkan 
\begin{verbatim}
self.opsi = webdriver.firefox.options.Options()
\end{verbatim}
\chapter{Pertemuan 5}

\section{Issues \#41}
Pada \textit{issues \#41} (\textit{method def ceknilai1(self):}) sebuah fungsi untuk melihat atau mengecek nilai semester ganjil 2018-2019
\begin{lstlisting}[language=Python]
def ceknilai1(self):
        self.opsi = Options()
        self.opsi.headless = False
        self.cap = webdriver.common.desired_capabilities.DesiredCapabilities().FIREFOX
        self.cap['marionette'] = True
        self.driver = webdriver.Firefox()
        self.driver.get('http://siap.poltekpos.ac.id/siap/besan.depan.php')
        self.driver.find_element_by_name('user_name').send_keys(self.npm)
        self.driver.find_element_by_name('user_pass').send_keys(self.paswd)
        self.driver.find_element_by_xpath('/html/body/table/tbody/tr[5]/td/table[1]/tbody/tr/td[2]/table[2]/tbody/tr[1]/td[2]/div/form/input[4]').click()
        self.driver.find_element_by_xpath('/html/body/table/tbody/tr[5]/td/table[1]/tbody/tr/td[1]/table[2]/tbody/tr[1]/td[2]/a[5]').click()
        self.driver.find_element_by_xpath('/html/body/table/tbody/tr[5]/td/table[3]/tbody/tr[1]/td[2]/p[1]/table/tbody/tr/td[3]/select/option[5]').click()
        self.driver.find_element_by_class_name('button').click()
\end{lstlisting}

\section{Issues \#42}
Pada \textit{issues \#42} ini merupakan sebuah fungsi untuk mengambil elemen atau \textit{find element} dari Nilai Mahasiswa dengan menggunakan \textit{xpath} 
\begin{lstlisting}[language=Python]
 self.driver.find_element_by_xpath('/html/body/table/tbody/tr[5]/td/table[1]/tbody/tr/td[1]/table[2]/tbody/tr[1]/td[2]/a[5]').click()
\end{lstlisting}

\section{Issues \#43}
Pada \textit{issues \#43} ini merupakan sebuah fungsi untuk mengambil elemen atau \textit{find element} dari Tahun Akademik Semester Ganjil 2018-2019
\begin{lstlisting}[language=Python]
/html/body/table/tbody/tr[5]/td/table[3]/tbody/tr[1]/td[2]/p[1]/table/tbody/tr/td[3]/select/option[5]
\end{lstlisting}


\section{Issues \#44}
Pada \textit{issues \#44} ini merupakan sebuah fungsi untuk mengambil elemen atau \textit{find element} dari tombol cari.
\begin{lstlisting}[language=Python]
self.driver.find_element_by_class_name('button').click()
\end{lstlisting}

\section{Issues \#45}
Pada \textit{issues \#45} (\textit{z = "Cek nilai semester 1"}) ini merupakan sebuah fungsi untuk \textit{print text} sesuai apa yang diucapkan dan dikenali oleh \textit{speech recognition}
\begin{lstlisting}[language=Python]
def speak(self):
        r= sr.Recognizer()
        with sr.Microphone() as source:
            print("SAY SOMETHING, PLEASE")
            audio = r.listen(source)
        try:
            print("TEXT : "+r.recognize_google(audio, language='id-ID'))
             z = "Cek nilai semester 1"
\end{lstlisting}

\section{Issues \#46}
Pada \textit{issues \#46} \textit{def ceknilai2(self):
} sebuah fungsi untuk melihat atau mengecek nilai semester genap 2018/2019
\begin{lstlisting}[language=Python]
 def ceknilai2(self):
        self.opsi = Options()
        self.opsi.headless = False
        self.cap = webdriver.common.desired_capabilities.DesiredCapabilities().FIREFOX
        self.cap['marionette'] = True
        self.driver = webdriver.Firefox()
        self.driver.get('http://siap.poltekpos.ac.id/siap/besan.depan.php')
        self.driver.find_element_by_name('user_name').send_keys(self.npm)
        self.driver.find_element_by_name('user_pass').send_keys(self.paswd)
        self.driver.find_element_by_xpath('/html/body/table/tbody/tr[5]/td/table[1]/tbody/tr/td[2]/table[2]/tbody/tr[1]/td[2]/div/form/input[4]').click()
        self.driver.find_element_by_xpath('/html/body/table/tbody/tr[5]/td/table[1]/tbody/tr/td[1]/table[2]/tbody/tr[1]/td[2]/a[5]').click()
        self.driver.find_element_by_xpath('/html/body/table/tbody/tr[5]/td/table[3]/tbody/tr[1]/td[2]/p[1]/table/tbody/tr/td[3]/select/option[4]').click()
        self.driver.find_element_by_class_name('button').click()
\end{lstlisting}

\section{Issues \#47}
Pada \textit{issues \#47} \textit{def ceknilai3(self):} sebuah fungsi untuk melihat atau mengecek nilai semester 3
\begin{lstlisting}[language=Python]
 def ceknilai3(self):
        self.opsi = Options()
        self.opsi.headless = False
        self.cap = webdriver.common.desired_capabilities.DesiredCapabilities().FIREFOX
        self.cap['marionette'] = True
        self.driver = webdriver.Firefox()
        self.driver.get('http://siap.poltekpos.ac.id/siap/besan.depan.php')
        self.driver.find_element_by_name('user_name').send_keys(self.npm)
        self.driver.find_element_by_name('user_pass').send_keys(self.paswd)
        self.driver.find_element_by_xpath('/html/body/table/tbody/tr[5]/td/table[1]/tbody/tr/td[2]/table[2]/tbody/tr[1]/td[2]/div/form/input[4]').click()
        self.driver.find_element_by_xpath('/html/body/table/tbody/tr[5]/td/table[1]/tbody/tr/td[1]/table[2]/tbody/tr[1]/td[2]/a[5]').click()
        self.driver.find_element_by_xpath('/html/body/table/tbody/tr[5]/td/table[3]/tbody/tr[1]/td[2]/p[1]/table/tbody/tr/td[3]/select/option[2]').click()
        self.driver.find_element_by_class_name('button').click()
\end{lstlisting}

\section{Issues \#48}
Pada \textit{issues \#48} \textit{def ceknilaipendek(self):}) sebuah fungsi untuk mengecek nilai semester pendek.
\begin{lstlisting}[language=Python]
 def ceknilaipendek(self):
        self.opsi = Options()
        self.opsi.headless = False
        self.cap = webdriver.common.desired_capabilities.DesiredCapabilities().FIREFOX
        self.cap['marionette'] = True
        self.driver = webdriver.Firefox()
        self.driver.get('http://siap.poltekpos.ac.id/siap/besan.depan.php')
        self.driver.find_element_by_name('user_name').send_keys(self.npm)
        self.driver.find_element_by_name('user_pass').send_keys(self.paswd)
        self.driver.find_element_by_xpath('/html/body/table/tbody/tr[5]/td/table[1]/tbody/tr/td[2]/table[2]/tbody/tr[1]/td[2]/div/form/input[4]').click()
        self.driver.find_element_by_xpath('/html/body/table/tbody/tr[5]/td/table[1]/tbody/tr/td[1]/table[2]/tbody/tr[1]/td[2]/a[5]').click()
        self.driver.find_element_by_xpath('/html/body/table/tbody/tr[5]/td/table[3]/tbody/tr[1]/td[2]/p[1]/table/tbody/tr/td[3]/select/option[3]').click()
        self.driver.find_element_by_class_name('button').click()
\end{lstlisting}

\section{Issues \#49}
Pada \textit{issues \#49} \textit{a = "Cek nilai semester 2"
} ini merupakan sebuah fungsi untuk \textit{print text} sesuai apa yang diucapkan dan dikenali oleh \textit{speech recognition}
\begin{lstlisting}[language=Python]
def speak(self):
        r= sr.Recognizer()
        with sr.Microphone() as source:
            print("SAY SOMETHING, PLEASE")
            audio = r.listen(source)
        try:
            print("TEXT : "+r.recognize_google(audio, language='id-ID'))
             a = "Cek nilai semester 2"
\end{lstlisting}

\section{Issues \#50}
Pada \textit{issues \#50} \textit{b = "Cek nilai semester 3"} ini merupakan sebuah fungsi untuk \textit{print text} sesuai apa yang diucapkan dan dikenali oleh \textit{speech recognition}
\begin{lstlisting}[language=Python]
def speak(self):
        r= sr.Recognizer()
        with sr.Microphone() as source:
            print("SAY SOMETHING, PLEASE")
            audio = r.listen(source)
        try:
            print("TEXT : "+r.recognize_google(audio, language='id-ID'))
             b = "Cek nilai semester 3"
\end{lstlisting}


\chapter{Pertemuan 6}

\section{Issues \#51}
Pada \textit{issues \#51} \textit{def kalenderganjil2019(self):}) sebuah fungsi untuk mengecek kalender ganjil 2019
\begin{lstlisting}[language=Python]
 def kalenderganjil2019(self):
        self.opsi = Options()
        self.opsi.headless = False
        self.cap = webdriver.common.desired_capabilities.DesiredCapabilities().FIREFOX
        self.cap['marionette'] = True
        self.driver = webdriver.Firefox()
        self.driver.get('http://siap.poltekpos.ac.id/siap/besan.depan.php')
        self.driver.find_element_by_name('user_name').send_keys(self.npm)
        self.driver.find_element_by_name('user_pass').send_keys(self.paswd)
        self.driver.find_element_by_xpath('/html/body/table/tbody/tr[5]/td/table[1]/tbody/tr/td[2]/table[2]/tbody/tr[1]/td[2]/div/form/input[4]').click()
        self.driver.find_element_by_xpath('/html/body/table/tbody/tr[5]/td/table[1]/tbody/tr/td[1]/table[2]/tbody/tr[1]/td[2]/a[6]').click()
        self.driver.find_element_by_class_name('textbox').click()
        self.driver.find_element_by_xpath('/html/body/table/tbody/tr[5]/td/table[3]/tbody/tr[1]/td[2]/p/table/tbody/tr[2]/td[2]/select/option[1]').click()
        self.driver.find_element_by_xpath('/html/body/table/tbody/tr[5]/td/table[3]/tbody/tr[1]/td[2]/p[1]/table/tbody/tr[4]/td[2]/select/option[2]').click()
        self.driver.find_element_by_xpath('/html/body/table/tbody/tr[5]/td/table[3]/tbody/tr[1]/td[2]/p/table/tbody/tr[5]/td/input').click()

\end{lstlisting}

\section{Issues \#52}
Pada \textit{issues \#52} ini merupakan fungsi untuk mengambil elemen atau \textit{find element} dari Kalender Akademik.
\begin{lstlisting}[language=Python]
self.driver.find_element_by_xpath('/html/body/table/tbody/tr[5]/td/table[1]/tbody/tr/td[1]/table[2]/tbody/tr[1]/td[2]/a[6]').click()
\end{lstlisting}

\section{Issues \#53}
Pada \textit{issues \#53} ini merupakan sebuah fungsi untuk mengambil elemen atau \textit{find element} dari \textit{class name textbox} yaitu elemen dari Tahun Akademik.
\begin{verbatim}
self.driver.find_element_by_class_name('textbox').click()
\end{verbatim}

\section{Issues \#54}
Pada \textit{issues \#54} ini merupakan sebuah fungsi mengambil elemen atau \textit{find element} \textit{by xpath}
\begin{lstlisting}[language=Python]
self.driver.find_element_by_xpath('/html/body/table/tbody/tr[5]/td/table[3]/tbody/tr[1]/td[2]/p/table/tbody/tr[2]/td[2]/select/option[1]').click()
\end{lstlisting}

\section{Issues \#55}
Pada \textit{issues \#55} ini merupakan sebuah fungsi untuk mengambil elemen atau \textit{find element by xpath} dari program studi 14- d4 Teknik Informatika
\begin{lstlisting}[language=Python]
self.driver.find_element_by_xpath('/html/body/table/tbody/tr[5]/td/table[3]/tbody/tr[1]/td[2]/p[1]/table/tbody/tr[4]/td[2]/select/option[2]').click()
\end{lstlisting}



\section{Issues \#56}
Pada \textit{issues \#56} fungsi ini digunakan untuk mengambil elemen atau \textit{find element by xpath} dari tombol tampilkan
\begin{lstlisting}[language=Python]
  self.driver.find_element_by_xpath('/html/body/table/tbody/tr[5]/td/table[3]/tbody/tr[1]/td[2]/p/table/tbody/tr[5]/td/input').click()
\end{lstlisting}


\section{Issues \#57}
Pada \textit{issues \#57}  fungsi ini dignakan untuk mengambil elemen atau \textit{find element by xpath} dari tahun akademik semester ganjil 2019/2020
\begin{lstlisting}[language=Python]
 self.driver.find_element_by_xpath('/html/body/table/tbody/tr[5]/td/table[3]/tbody/tr[1]/td[2]/p/table/tbody/tr[2]/td[2]/select/option[2]').click()
\end{lstlisting}

\section{Issues \#58}
Pada \textit{issues \#58} fungsi ini digunakan untuk mengambil elemen atau \textit{find element by xpath} dari tahun akademik semester antara 2018/2019
\begin{lstlisting}[language=Python]
self.driver.find_element_by_xpath('/html/body/table/tbody/tr[5]/td/table[3]/tbody/tr[1]/td[2]/p/table/tbody/tr[2]/td[2]/select/option[3]').click()
\end{lstlisting}

\section{Issues \#59}
Pada \textit{issues \#59}  fungsi ini digunakan untk mengambil elemen atau \textit{find element by xpath} dari tahun akademik semster genap 2018/2019
\begin{lstlisting}[language=Python]
self.driver.find_element_by_xpath('/html/body/table/tbody/tr[5]/td/table[3]/tbody/tr[1]/td[2]/p/table/tbody/tr[2]/td[2]/select/option[4]').click()
\end{lstlisting}

\section{Issues \#60}
Pada \textit{issues \#60}  fungsi ini digunakan untuk mengambil elemen atau \textit{find element by xpath} dari tahun akademik semester ganjil 2018/2019
\begin{lstlisting}[language=Python]
self.driver.find_element_by_xpath('/html/body/table/tbody/tr[5]/td/table[3]/tbody/tr[1]/td[2]/p/table/tbody/tr[2]/td[2]/select/option[5]').click()
\end{lstlisting}





\chapter{Pertemuan 7}

\section{Issues \#61}
Pada \textit{issues \#61} ini adalah sebuah fungsi untuk mengeklik  Jadwal Kuliah dengan mengambil elemen atau \textit{find element by link text} si Jadwal Kuliah
\begin{lstlisting}[language=Python]
	self.driver.find_element_by_link_text('Jadwal Kuliah').click()
\end{lstlisting}

\section{Issues \#62}
Pada \textit{issues \#62} ini adalah sebuah fungsi untuk mengambil elemen atau \textit{find element} tahun akademik semester ganjil 2019/2020
\begin{lstlisting}[language=Python]
	  self.driver.find_element_by_xpath('/html/body/table/tbody/tr[5]/td/table[3]/tbody/tr[1]/td[2]/p[1]/table/tbody/tr/td[3]/select/option[2]').click()
\end{lstlisting}

\section{Issues \#63}
Pada \textit{issues \#63} ini adalah sebuah fungsi untuk mengambil elemen atau \textit{find element by class name} dari si tampilkan dengan \textit{class} nya namanya \textit{button}
\begin{lstlisting}[language=Python]
self.driver.find_elements_by_class_name('button').click()
\end{lstlisting}


\section{Issues \#64}
Pada \textit{issues \#64} ini adalah sebuah fungsi untuk mengambil elemen atau \textit{find element by xpath} dari tahun akademik semester genap 2018/2019
\begin{lstlisting}[language=Python]
self.driver.find_element_by_xpath('/html/body/table/tbody/tr[5]/td/table[3]/tbody/tr[1]/td[2]/p[1]/table/tbody/tr/td[3]/select/option[3]').click()
\end{lstlisting}

\section{Issues \#65}
Pada \textit{issues \#65} ini adalah sebuah fungsi untuk mengambil elemen atau \textit{find element by xpath} dari tahun akademik semester ganjil 2018-2019
\begin{lstlisting}[language=Python]
self.driver.find_element_by_xpath('/html/body/table/tbody/tr[5]/td/table[3]/tbody/tr[1]/td[2]/p[1]/table/tbody/tr/td[3]/select/option[4]').click()
\end{lstlisting}

\section{Issues \#66}
Pada \textit{issues \#66} ini adalah sebuah fungsi untuk jika audio menangkap suara seperti \textit{variable} misalnya j maka akan melakukan \textit{method self.jadwalsmt1()}
\begin{lstlisting}[language=Python]
if (r.recognize_google(audio, language='id-ID')) == j:
    self.jadwalsmt1()
\end{lstlisting}

\section{Issues \#67}
Pada \textit{issues \#67} ini adalah sebuah fungsi untuk jika audio dari \textit{recognize google} menangkap suara seperti \textit{variable} misalnya k maka akan melakukan \textit{method jadwalsmt2}
\begin{lstlisting}[language=Python]
if (r.recognize_google(audio, language='id-ID')) == k:
    self.jadwalsmt2()
\end{lstlisting}

\section{Issues \#68}
Pada \textit{issues \#68} ini adalah sebuah fungsi untuk jika audio dari \textit{recognize google} menangkap suara seperti \textit{variable} misalnya l maka akan melakukan method jadwalsmt3()
\begin{lstlisting}[language=Python]
if (r.recognize_google(audio, language='id-ID')) == l:
    self.jadwalsmt3()
\end{lstlisting}


\section{Issues \#69}
Pada \textit{issues \#69} \textit{[WinError 10054] An existing connection was forcibly closed by the remote host} sebuah \textit{error} yang disebabkan oleh Memberitahu Anda bahwa sudah ada koneksi aktif di tempat kesalahan di mana Anda mungkin mencoba untuk membuka yang kedua. \\
cara mengatasinya yaitu putuskan sambungan wifi, \textit{restart} laptop, koneksikan wifi yang akan digunakan secara manual saja

\section{Issues \#70}
Pada \textit{issues \#70} (\textit{'list' object has no attribute 'click'
error}) disebabkan dikarenakan tidak dapat mengambil atau \textit{find element by link text} pada Jadwal Kuliah, karena salah penulisan pada Jadwal kuliah.

\chapter{Pertemuan 8}

\section{Issues \#71}
Pada \textit{issues \#71} (\textit{from time import sleep}) adalah sebuah fungsi untuk menambahkan jeda waktu dengan \textit{import sleep}
\begin{lstlisting}[language=Python]
from time import sleep
\end{lstlisting}


\section{Issues \#72}
Pada \textit{issues \#72} adalah \textit{sleep(2)} menambahkan jeda waktu selama 2 detik

\section{Issues \#73}
Pada \textit{issues \#73} ini merupakan fungsi untuk mengambil elemen atau \textit{element by xpath} dari Ubah Profil
\begin{lstlisting}[language=Python]
  self.driver.find_element_by_xpath('/html/body/table/tbody/tr[5]/td/table[1]/tbody/tr/td[1]/table[2]/tbody/tr[1]/td[2]/a[1]').click()
\end{lstlisting}

\section{Issues \#74}
Pada \textit{issues \#74} ini merupakan fungsi untuk mengambil elemen atau \textit{find element by xpath} dari \textit{input name} dari \textit{class textbox} dan dengan \textit{name}nya \textit{Handphone} dengan menggunakan \textit{variable} nomorsaya
\begin{lstlisting}[language=Python]
self.nomorsaya = self.driver.find_element_by_xpath("//input[@name='Handphone']")
\end{lstlisting}

\section{Issues \#75}
Pada \textit{issues \#75} ini merupakan fungsi untuk mengambil elemen atau \textit{get attribute value} dengan \textit{variable} nomorsayas
\begin{lstlisting}[language=Python]
self.nomorsayas = self.nomorsaya.get_attribute('value')
\end{lstlisting}


\section{Issues \#76}
Pada \textit{issues \#76} ini merupakan fungsi untuk mengambil elemen atau \textit{find element by link text} Orang Tua
\begin{lstlisting}[language=Python]
self.driver.find_element_by_link_text('Orang Tua').click()
\end{lstlisting}


\section{Issues \#77}
Pada \textit{issues \#77} \textit{def nomerortu(self):} ini merupakan fungsi untuk mengecek dan mengambil \textit{value} dari nomer orang tua
\begin{lstlisting}[language=Python]
 def nomerortu(self):
        self.opsi = Options()
        self.opsi.headless = False
        self.cap = webdriver.common.desired_capabilities.DesiredCapabilities().FIREFOX
        self.cap['marionette'] = True
        self.driver = webdriver.Firefox()
        self.driver.get('http://siap.poltekpos.ac.id/siap/besan.depan.php')
        self.driver.find_element_by_name('user_name').send_keys(self.npm)
        self.driver.find_element_by_name('user_pass').send_keys(self.paswd)
        self.driver.find_element_by_xpath('/html/body/table/tbody/tr[5]/td/table[1]/tbody/tr/td[2]/table[2]/tbody/tr[1]/td[2]/div/form/input[4]').click()
        self.driver.find_element_by_xpath('/html/body/table/tbody/tr[5]/td/table[1]/tbody/tr/td[1]/table[2]/tbody/tr[1]/td[2]/a[1]').click()
        sleep(2)
        self.driver.find_element_by_link_text('Orang Tua').click()
        sleep(2)
        self.nomor = self.driver.find_element_by_xpath("//input[@name='HandphoneOrtu']")
        self.nomornya = self.nomor.get_attribute('value')
        print(self.nomornya)
\end{lstlisting}

\section{Issues \#78}
Pada \textit{issues \#78} \textit{def emailortu(self):}ini merupakan fungsi untuk mengecek atau mengambil \textit{value} dari email orang tua
\begin{lstlisting}[language=Python]
 def emailortu(self):
        self.opsi = Options()
        self.opsi.headless = False
        self.cap = webdriver.common.desired_capabilities.DesiredCapabilities().FIREFOX
        self.cap['marionette'] = True
        self.driver = webdriver.Firefox()
        self.driver.get('http://siap.poltekpos.ac.id/siap/besan.depan.php')
        self.driver.find_element_by_name('user_name').send_keys(self.npm)
        self.driver.find_element_by_name('user_pass').send_keys(self.paswd)
        self.driver.find_element_by_xpath('/html/body/table/tbody/tr[5]/td/table[1]/tbody/tr/td[2]/table[2]/tbody/tr[1]/td[2]/div/form/input[4]').click()
        self.driver.find_element_by_link_text('Ubah Profil').click()
        self.driver.find_element_by_link_text('Orang Tua').click()
        sleep(2)
        self.emailortu = self.driver.find_element_by_xpath("//input[@name='EmailOrtu']")
        sleep(2)
        self.emailortus = self.emailortu.get_attribute('value')
        print(self.emailortus)
\end{lstlisting}


\section{Issues \#79}
Pada \textit{issues \#79} \textit{print(self.nomorsayas)} ini merupakan fungsi untuk \textit{print variable} dari nomorsayas

\section{Issues \#80}
Pada \textit{issues \#80} ini meruapakan fungsi untuk \textit{print text} sesuai yang diucapkan dan dikenali oleh \textit{speech recognition}dan sebuah fungsi untuk jika audio dari \textit{recognize google} menangkap suara seperti \textit{variable} misalnya p maka akan melakukan \textit{method emailortu}
%next line adds the Bibliography to the contents page
%\addcontentsline{toc}{chapter}{Bibliography}
%uncomment next line to change bibliography name to references
%\renewcommand{\bibname}{References}
%\bibliography{references}        %use a bibtex bibliography file refs.bib
%\bibliographystyle{plain}  %use the plain bibliography style

\end{document}

