\chapter{Pertemuan 1}

\section{Issues \#1}
Pada \textit{issues \#1} (\textit{Hardcoded text}) Keterangan dan pemechan masalah : ini hanya peringatan, fungsi tetap bekerja sebagai mana mestinya. direkomendasikan untuk mendefine string "text" di file string.xml. alasannya agar membuat perubahan di masa depan lebih sederhana dan lebih mudah untuk referensi di beberapa halaman.
\\lokasi
\begin{verbatim}
  <Button
  android:id="@+id/buttonScan"
  android:layout_width="match_parent"
  android:layout_height="wrap_content"
  android:layout_alignParentBottom="true"
  android:text="Scan QR Code" />       
\end{verbatim}
sehingga solusinya didapatkan dengan cara mendownload packages python pyAudio yang sesuai dengan versi python yang digunakan. Setelah itu install melalui cmd dengan mengetikkan \begin{verbatim}
  <resources>
  <string name="app_name">QRCodeScanner</string>
  </resources>
\end{verbatim}
Pada merupakan error yang terjadi di saat hendak melakukan instalasi PyAudio melalui pip install PyAudio\begin{verbatim}
  <TextView
    android:layout_width="match_parent"
    android:layout_height="wrap_content"
    android:text="@string/string_name"/>
\end{verbatim}
Pada merupakan error yang terjadi di saat hendak melakukan instalasi PyAudio melalui pip install PyAudio\begin{verbatim}
  <resources>
    <string name="app_name">QRCodeScanner</string>
    <string name="scan_qr_code">Scan Qr Code</string>
  </resources>
\end{verbatim}

\section{Issues \#2}
Pada \textit{issues \#2} (\textit{MissingPrefix}) yaitu Error yang terjadi ketika xxxx.
\\lokasi :
\begin{verbatim}
  <Button
  android:id="@+id/buttonScan"
  android:layout_width="match_parent"
  android:layout_height="wrap_content"
  android:layout_alignParentBottom="true"
  android:text="Scan QR Code" />       
\end{verbatim}
Perbaikan kode progam :\begin{verbatim}
  <TextView
    android:layout_width="match_parent"
    android:layout_height="wrap_content"
    android:text="@string/string_name"/>
\end{verbatim}

\section{Issues \#3}
Pada \textit{issues \#2} (\textit{Jenis Error : identifier expected}) yaitu Error yang terjadi ketika xxxx.
\\lokasi :
\begin{verbatim}
  protected void onPostExecute(s) {
     super.onPostExecute(s);
     loading.dismiss();
     Toast.makeText(MainActivity.this,s,Toast.LENGTH_LONG).show();
  }
\end{verbatim}
Perbaikan kode progam :\begin{verbatim}
  protected void onPostExecute(String s) {
     super.onPostExecute(s);
     loading.dismiss();
     Toast.makeText(MainActivity.this,s,Toast.LENGTH_LONG).show();
  }
\end{verbatim}

\section{Issues \#4}
Pada \textit{issues \#2} (\textit{Jenis Error : variable ‘x’ is already defined in scope})
\\lokasi :
\begin{verbatim}
  RequestHandler rh = new RequestHandler();
  String rh = rh.sendPostRequest(konfigurasi.URL_ADD, params);
  return rh;
\end{verbatim}
Keterangan dan splusi :
\\tidak dapat memanggil variabel sendidiri dan membuat variabel lokal dengan nama yang sama di java. solusi yang dapat dilakukan adalah mengganti nama salah variable
Perbaikan kode progam :\begin{verbatim}
  protected void onPostExecute(String s) {
     super.onPostExecute(s);
     loading.dismiss();
     Toast.makeText(MainActivity.this,s,Toast.LENGTH_LONG).show();
  }
\end{verbatim}

\section{Issues \#5}
Pada \textit{issues \#2} (\textit{Jenis Error : Incompatible Types})
\\lokasi :
\begin{verbatim}
  buttonScan = (“Button”) findViewById(R.id.buttonScan);
  buttonAdd = (“Button”) findViewById(R.id.buttonAdd);
  buttonTabel = (“Button”) findViewById(R.id.buttonTabel);
\end{verbatim}
Keterangan dan pemecahan masalah :
\\ini merupakan sebuah error pada MainActivity.java karena penggunaan dalam syntak salah yaitu dengan menggunakan tanda kutip pada kode Button. adalah mengganti nama salah variable
Perbaikan kode progam :\begin{verbatim}
  buttonScan = (Button) findViewById(R.id.buttonScan);
  buttonAdd = (Button) findViewById(R.id.buttonAdd);
  buttonTabel = (Button) findViewById(R.id.buttonTabel);
\end{verbatim}

\section{Issues \#6}
Pada \textit{issues \#2} (\textit{warning Field can be converted to a local variable Lokasi File MainActivity.java})
\\lokasi :
Keterangan dan pemecahan masalah :
\\ini merupakan sebuah warning atau peringatan pada MainActivity.java karena
pada IntentIntegrator kelebihan dalam bidang kelasnya dan jika kelebihan bisa diganti dengan variable lokal.
\end{verbatim}

\section{Issues \#7}
Pada \textit{issues \#2} (\textit{Local variable 'res' is redundant less Lokasi File MainActivity.java})
\\lokasi :
\begin{verbatim}
String res = rh.sendPostRequest(konfigurasi.URL_ADD, params);
\end{verbatim}
\\ini merupakan sebuah warning atau peringatan pada MainActivity.java, warning tersebut meliputi; variabel yang ditangkap meliputi variabel lokal yang segera dikembalikan, variabel lokal yang segera ditugaskan ke variabel lain dan kemudian tidak digunakan, dan variabel lokal yang selalu memiliki nilai yang sama dengan variabel atau parameter lokal lainnya.


\section{Issues \#8}
Pada \textit{issues \#2} (\textit{Cannot resolve symbol ‘parent’})
\\lokasi :
protected void onActivityResult(int requestCode, int resultCode, Intent data) {
IntentResult result = IntentIntegrator.parseActivityResult(requestCode, resultCode, data);
if (result != null)
\\Keterangan dan Solusi:
Library belum diimport solusinya dengan cara mengimport library jenis zxying:
\begin{verbatim}
import com.google.zxing.integration.android.IntentIntegrator;
import com.google.zxing.integration.android.IntentResult;
\end{verbatim}

\section{Issues \#9}
Pada \textit{issues \#2} (\textit{taticFieldLeak})
\\lokasi : Lokasi File MainActivity.java
protected void onActivityResult(int requestCode, int resultCode, Intent data) {
IntentResult result = IntentIntegrator.parseActivityResult(requestCode, resultCode, data);
if (result != null)
\end{verbatim}
\\Keterangan dan pemecahan masalah:
AsyncTask sebagai class non static inner yang akan menerima reference dari parent class. untuk memperbaikinya teruskan aktivity yang lemah ke constructor AsyncTask. kemudian kamu dapat memanggil intent code di onPostExecute.

\section{Issues \#10}
Pada \textit{issues \#10} (\textit{XXXXXXXXXXX)}) fungsi yang digunakan untuk memberikan \textit{Exception} apabila program gagal mendeteksi suara atau terjadi kesalahan atau \textit{error} yang kemudian akan dicetak \textit{error} tersebut dan mencetak "TIME OVER, THANK YOU"