\chapter{Pertemuan 3}
\section{Issues \#21}
Pada \textit{issues \#21} (\textit{class Apem(object):}) ialah pembuatan class sehingga program dapat dijalankan secara \textit{Object-Oriented-Programming} / OOP 
\begin{verbatim}
class Apem(object): 
\end{verbatim}
\section{Issues \#22}
Pada \textit{issues \#22} (\textit{def \_\_init\_\_(self, filname): self.filname = filname}) pembuatan constructor dengan parameter self dan filname
\section{Issues \#23}
Pada \textit{issues \#23} (\textit{def masuk(self):}) ialah fungsi yang berisikan method yang akan menampilkan tampilan halaman sistem akademik siap di Politeknik Pos Indonesia
\begin{verbatim}
def masuk(self): 
        self.opsi = Options()
        self.opsi.headless = False
        self.cap = webdriver.common.desired_capabilities.DesiredCapabilities().FIREFOX
        self.cap['marionette'] = True
        self.driver= webdriver.Firefox()
        self.driver.get('http://siap.poltekpos.ac.id/')
\end{verbatim}
\section{Issues \#24}
Pada \textit{issues \#24} (\textit{def login(self):}) fungsi yang berisikan method yang akan menampilkan tampilan halaman login ke sistem akademik siap di Politeknik Pos Indonesia
\begin{verbatim}
 def login(self): 
        self.opsi = Options()
        self.opsi.headless = False
        self.cap = webdriver.common.desired_capabilities.DesiredCapabilities().FIREFOX
        self.cap['marionette'] = True
        self.driver= webdriver.Firefox()
        self.driver.get('http://siap.poltekpos.ac.id/siap/besan.depan.php')
        self.driver.find_element_by_name('user_name').send_keys(self.npm)
        self.driver.find_element_by_name('user_pass').send_keys(self.paswd)
        self.driver.find_element_by_xpath('/html/body/table/tbody/tr[5]/td/table[1]/tbody/tr/td[2]/table[2]/tbody/tr[1]/td[2]/div/form/input[4]').click()
\end{verbatim}
\section{Issues \#25}
Pada \textit{issues \#25} (\textit{self.driver.find\_element\_by\_xpath('/html/body/table/tbody/tr[5]/td/table[1]/tbody/tr/td[2]/table[2]/tbody/tr[1]/td[2]/div/form/input[4]').click()}) merupakan variabel yang memungkinkan driver untuk mendapatkan element pada button login melalui xpath kemudian program akan secara otomatis melakukan method click pada button tersebut
\section{Issues \#26}
Pada \textit{issues \#26} (\textit{def speak(self):}) fungsi yang dibuat untuk memberikan method yang terjadi ketika user mengucapkan sesuatu ke program kemudian program menangkap audio yang diterima dengan menggunakan \textit{try and except}
\begin{verbatim}
def speak(self):
        r= sr.Recognizer()
        with sr.Microphone() as source:
            print("SAY SOMETHING, PLEASE")
            audio = r.listen(source)
        try:
            print("TEXT : "+r.recognize_google(audio, language='id-ID'))
            x = "siap"
            y = "login siap"
            z = "Cek nilai semester 1"
            a = "Cek nilai semester 2"
            b = "Cek nilai semester 3"
            c = "Cek nilai semester pendek"
            d = "kalender akademik ganjil 2017"
            e = "kalender akademik ganjil 2018"
            f = "kalender akademik ganjil 2019"
            g = "kalender akademik genap 2017"
            h = "kalender akademik genap 2018"
            i = "kalender akademik genap 2019"
            j = "jadwal semester 3"
            k = "jadwal semester 2"
            l = "jadwal semester 1"
            m = "nomor saya"
            n = "alamat saya"
            o = "nomor ortu"
            p = "mail ortu"
            if (r.recognize_google(audio, language='id-ID')) == x:
                self.masuk()
            if (r.recognize_google(audio, language='id-ID')) == y:
                self.login()
            if (r.recognize_google(audio, language='id-ID')) == z:
                self.ceknilai1()
            if (r.recognize_google(audio, language='id-ID')) == a:
                self.ceknilai2()
            if (r.recognize_google(audio, language='id-ID')) == b:
                self.ceknilai3()
            if (r.recognize_google(audio, language='id-ID')) == c:
                self.ceknilaipendek()
            if (r.recognize_google(audio, language='id-ID')) == d:
                self.kalenderganjil2017()
            if (r.recognize_google(audio, language='id-ID')) == e:
                self.kalenderganjil2018()
            if (r.recognize_google(audio, language='id-ID')) == f:
                self.kalenderganjil2019()
            if (r.recognize_google(audio, language='id-ID')) == g:
                self.kalendergenap2017()
            if (r.recognize_google(audio, language='id-ID')) == h:
                self.kalendergenap2018()
            if (r.recognize_google(audio, language='id-ID')) == i:
                self.kalendergenap2019()
            if (r.recognize_google(audio, language='id-ID')) == j:
                self.jadwalsmt1()
            if (r.recognize_google(audio, language='id-ID')) == k:
                self.jadwalsmt2()
            if (r.recognize_google(audio, language='id-ID')) == l:
                self.jadwalsmt3()
            if (r.recognize_google(audio, language='id-ID')) == m:
                self.nomerhp()
            if (r.recognize_google(audio, language='id-ID')) == n:
                self.alamatsiswa()
            if (r.recognize_google(audio, language='id-ID')) == o:
                self.nomerortu()
            if (r.recognize_google(audio, language='id-ID')) == p:
                self.emailortu()
\end{verbatim}
\section{Issues \#27}
Pada \textit{issues \#27} (\textit{print("Time is over, thanks")}) merupakan method berupa mencetak "Time is over, thanks" yang akan dijalankan ketika program telah selesai dieksekusi
\begin{verbatim}
print("Time is over, thanks")
\end{verbatim}
\section{Issues \#28}
Pada \textit{issues \#28} (\textit{import apem}) merupakan perintah yang digunakan untuk mengimportkan file apem
\begin{verbatim}
import apem
\end{verbatim}
\section{Issues \#29}
Pada \textit{issues \#29} (\textit{speech = apem.Apem(object)}) merupakan variabel yang berisikan perintah untuk penginisiasian objek dari file apem di \textit{class} Apem.
\begin{verbatim}
speech = apem.Apem(object)
\end{verbatim}
\section{Issues \#30}
Pada \textit{issues \#30} (\textit{speech.speak()}) merupakan perintah yang digunakan untuk menjalankan method pada fungsi speak() yang telah dibuat pada class Apem sebelumnya.
\begin{verbatim}
speech.speak()
\end{verbatim}