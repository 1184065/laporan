\chapter{Pertemuan 2}

\section{Issues \#11}
Pada \textit{issues \#11} (\textit{from selenium import webdriver}) merupakan fungsi yang digunakan untuk mengimportkan webdriver dari modul selenium
\begin{verbatim}
from selenium import webdriver
\end{verbatim}
\section{Issues \#12}
Pada \textit{issues \#12} (\textit{from selenium.webdriver.firefox.options import Options}) 
digunakan untuk mengimport Options dari webdriver firefox dari modul selenium
\begin{verbatim}
from selenium.webdriver.firefox.options import Options
\end{verbatim}
\section{Issues \#13}
Pada \textit{issues \#13} (\textit{opsi = Options()}) 
untuk membuat variable opsi dengan value Options() dari modul selenium yang telah diinstallkan pada program di PC kita
\begin{verbatim}
opsi = Options()
\end{verbatim}
\section{Issues \#14}
Pada \textit{issues \#14} (\textit{opsi = webdriver.firefox.options.Options()}) ialah variabel yang digunakan untuk mengambil value Options() dari webdriver firefox yang digunakan
\section{Issues \#15}
Pada \textit{issues \#15} (\textit{opsi.headless = False}) merupakan kode yang digunakan untuk menonaktifkan \textit{headless mode} pada firefox browser yang kita gunakan
\begin{verbatim}
opsi.headless = False
\end{verbatim}
\section{Issues \#16}
Pada \textit{issues \#16} (\textit{cap = webdriver.common.desired\_capabilities.DesiredCapabilities().FIREFOX}) variabel yang berfungsi untuk memberitahukan pada webdriver environtment mana yang akan digunakan pada \textit{script}.
\section{Issues \#17}
Pada \textit{issues \#17} (\textit{cap['marionette'] = True}) 
marionette digunakan untuk driver otomatisasi pada gecko mozilla firefox
\begin{verbatim}
cap['marionette'] = True
\end{verbatim}
\section{Issues \#18}
Pada \textit{issues \#18} (\textit{fungsionalitas driver = webdriver.Firefox()}) merupakan variabel yang digunakan untuk memulai suatu webdriver berbasiskan Firefox
\begin{verbatim}
driver = webdriver.Firefox()}
\end{verbatim}
\section{Issues \#19}
Pada \textit{issues \#19} (\textit{fungsionalitas driver.get("https://siap.poltekpos.ac.id")}) 
berfungsi untuk mendapatkan akses ke situs yang dituju melalui driver yang digunakan
\begin{verbatim}
driver.get("https://siap.poltekpos.ac.id")
\end{verbatim}
\section{Issues \#20}
Pada \textit{issues \#20} (\textit{fungsionalitas driver.find\_element\_by\_name('user\_name').send\_keys("1184030")}) digunakan untuk memungkinkan driver mendapatkan element username dengan melalui nama dan kemudian mengirimkan inputannya
\begin{verbatim}
self.driver.find_element_by_name('user_name').send_keys("1184030")
\end{verbatim}