\chapter{Pertemuan 5}

\section{Issues \#41}
Pada \textit{issues \#41} (\textit{method def ceknilai1(self):}) sebuah fungsi untuk melihat atau mengecek nilai semester ganjil 2018-2019
\begin{lstlisting}[language=Python]
def ceknilai1(self):
        self.opsi = Options()
        self.opsi.headless = False
        self.cap = webdriver.common.desired_capabilities.DesiredCapabilities().FIREFOX
        self.cap['marionette'] = True
        self.driver = webdriver.Firefox()
        self.driver.get('http://siap.poltekpos.ac.id/siap/besan.depan.php')
        self.driver.find_element_by_name('user_name').send_keys(self.npm)
        self.driver.find_element_by_name('user_pass').send_keys(self.paswd)
        self.driver.find_element_by_xpath('/html/body/table/tbody/tr[5]/td/table[1]/tbody/tr/td[2]/table[2]/tbody/tr[1]/td[2]/div/form/input[4]').click()
        self.driver.find_element_by_xpath('/html/body/table/tbody/tr[5]/td/table[1]/tbody/tr/td[1]/table[2]/tbody/tr[1]/td[2]/a[5]').click()
        self.driver.find_element_by_xpath('/html/body/table/tbody/tr[5]/td/table[3]/tbody/tr[1]/td[2]/p[1]/table/tbody/tr/td[3]/select/option[5]').click()
        self.driver.find_element_by_class_name('button').click()
\end{lstlisting}

\section{Issues \#42}
Pada \textit{issues \#42} ini merupakan sebuah fungsi untuk mengambil elemen atau \textit{find element} dari Nilai Mahasiswa dengan menggunakan \textit{xpath} 
\begin{lstlisting}[language=Python]
 self.driver.find_element_by_xpath('/html/body/table/tbody/tr[5]/td/table[1]/tbody/tr/td[1]/table[2]/tbody/tr[1]/td[2]/a[5]').click()
\end{lstlisting}

\section{Issues \#43}
Pada \textit{issues \#43} ini merupakan sebuah fungsi untuk mengambil elemen atau \textit{find element} dari Tahun Akademik Semester Ganjil 2018-2019
\begin{lstlisting}[language=Python]
/html/body/table/tbody/tr[5]/td/table[3]/tbody/tr[1]/td[2]/p[1]/table/tbody/tr/td[3]/select/option[5]
\end{lstlisting}


\section{Issues \#44}
Pada \textit{issues \#44} ini merupakan sebuah fungsi untuk mengambil elemen atau \textit{find element} dari tombol cari.
\begin{lstlisting}[language=Python]
self.driver.find_element_by_class_name('button').click()
\end{lstlisting}

\section{Issues \#45}
Pada \textit{issues \#45} (\textit{z = "Cek nilai semester 1"}) ini merupakan sebuah fungsi untuk \textit{print text} sesuai apa yang diucapkan dan dikenali oleh \textit{speech recognition}
\begin{lstlisting}[language=Python]
def speak(self):
        r= sr.Recognizer()
        with sr.Microphone() as source:
            print("SAY SOMETHING, PLEASE")
            audio = r.listen(source)
        try:
            print("TEXT : "+r.recognize_google(audio, language='id-ID'))
             z = "Cek nilai semester 1"
\end{lstlisting}

\section{Issues \#46}
Pada \textit{issues \#46} \textit{def ceknilai2(self):
} sebuah fungsi untuk melihat atau mengecek nilai semester genap 2018/2019
\begin{lstlisting}[language=Python]
 def ceknilai2(self):
        self.opsi = Options()
        self.opsi.headless = False
        self.cap = webdriver.common.desired_capabilities.DesiredCapabilities().FIREFOX
        self.cap['marionette'] = True
        self.driver = webdriver.Firefox()
        self.driver.get('http://siap.poltekpos.ac.id/siap/besan.depan.php')
        self.driver.find_element_by_name('user_name').send_keys(self.npm)
        self.driver.find_element_by_name('user_pass').send_keys(self.paswd)
        self.driver.find_element_by_xpath('/html/body/table/tbody/tr[5]/td/table[1]/tbody/tr/td[2]/table[2]/tbody/tr[1]/td[2]/div/form/input[4]').click()
        self.driver.find_element_by_xpath('/html/body/table/tbody/tr[5]/td/table[1]/tbody/tr/td[1]/table[2]/tbody/tr[1]/td[2]/a[5]').click()
        self.driver.find_element_by_xpath('/html/body/table/tbody/tr[5]/td/table[3]/tbody/tr[1]/td[2]/p[1]/table/tbody/tr/td[3]/select/option[4]').click()
        self.driver.find_element_by_class_name('button').click()
\end{lstlisting}

\section{Issues \#47}
Pada \textit{issues \#47} \textit{def ceknilai3(self):} sebuah fungsi untuk melihat atau mengecek nilai semester 3
\begin{lstlisting}[language=Python]
 def ceknilai3(self):
        self.opsi = Options()
        self.opsi.headless = False
        self.cap = webdriver.common.desired_capabilities.DesiredCapabilities().FIREFOX
        self.cap['marionette'] = True
        self.driver = webdriver.Firefox()
        self.driver.get('http://siap.poltekpos.ac.id/siap/besan.depan.php')
        self.driver.find_element_by_name('user_name').send_keys(self.npm)
        self.driver.find_element_by_name('user_pass').send_keys(self.paswd)
        self.driver.find_element_by_xpath('/html/body/table/tbody/tr[5]/td/table[1]/tbody/tr/td[2]/table[2]/tbody/tr[1]/td[2]/div/form/input[4]').click()
        self.driver.find_element_by_xpath('/html/body/table/tbody/tr[5]/td/table[1]/tbody/tr/td[1]/table[2]/tbody/tr[1]/td[2]/a[5]').click()
        self.driver.find_element_by_xpath('/html/body/table/tbody/tr[5]/td/table[3]/tbody/tr[1]/td[2]/p[1]/table/tbody/tr/td[3]/select/option[2]').click()
        self.driver.find_element_by_class_name('button').click()
\end{lstlisting}

\section{Issues \#48}
Pada \textit{issues \#48} \textit{def ceknilaipendek(self):}) sebuah fungsi untuk mengecek nilai semester pendek.
\begin{lstlisting}[language=Python]
 def ceknilaipendek(self):
        self.opsi = Options()
        self.opsi.headless = False
        self.cap = webdriver.common.desired_capabilities.DesiredCapabilities().FIREFOX
        self.cap['marionette'] = True
        self.driver = webdriver.Firefox()
        self.driver.get('http://siap.poltekpos.ac.id/siap/besan.depan.php')
        self.driver.find_element_by_name('user_name').send_keys(self.npm)
        self.driver.find_element_by_name('user_pass').send_keys(self.paswd)
        self.driver.find_element_by_xpath('/html/body/table/tbody/tr[5]/td/table[1]/tbody/tr/td[2]/table[2]/tbody/tr[1]/td[2]/div/form/input[4]').click()
        self.driver.find_element_by_xpath('/html/body/table/tbody/tr[5]/td/table[1]/tbody/tr/td[1]/table[2]/tbody/tr[1]/td[2]/a[5]').click()
        self.driver.find_element_by_xpath('/html/body/table/tbody/tr[5]/td/table[3]/tbody/tr[1]/td[2]/p[1]/table/tbody/tr/td[3]/select/option[3]').click()
        self.driver.find_element_by_class_name('button').click()
\end{lstlisting}

\section{Issues \#49}
Pada \textit{issues \#49} \textit{a = "Cek nilai semester 2"
} ini merupakan sebuah fungsi untuk \textit{print text} sesuai apa yang diucapkan dan dikenali oleh \textit{speech recognition}
\begin{lstlisting}[language=Python]
def speak(self):
        r= sr.Recognizer()
        with sr.Microphone() as source:
            print("SAY SOMETHING, PLEASE")
            audio = r.listen(source)
        try:
            print("TEXT : "+r.recognize_google(audio, language='id-ID'))
             a = "Cek nilai semester 2"
\end{lstlisting}

\section{Issues \#50}
Pada \textit{issues \#50} \textit{b = "Cek nilai semester 3"} ini merupakan sebuah fungsi untuk \textit{print text} sesuai apa yang diucapkan dan dikenali oleh \textit{speech recognition}
\begin{lstlisting}[language=Python]
def speak(self):
        r= sr.Recognizer()
        with sr.Microphone() as source:
            print("SAY SOMETHING, PLEASE")
            audio = r.listen(source)
        try:
            print("TEXT : "+r.recognize_google(audio, language='id-ID'))
             b = "Cek nilai semester 3"
\end{lstlisting}

